\chapter{Fazit}

Die Umsetzung des Projekts \textit{Pepper VR – Teleoperation eines humanoiden Roboters auf Basis der Analyse menschlicher Bewegung} stellte eine anspruchsvolle Herausforderung dar, die tiefgreifende Kenntnisse in verschiedenen Bereichen der Informatik und Robotik erforderte. Trotz der ambitionierten Ziele, die Verbindung zwischen der MetaQuest 3 VR-Brille und dem humanoiden Roboter Pepper herzustellen und die Steuerung von Pepper durch die VR-Controller zu ermöglichen, stießen wir auf mehrere technische und organisatorische Hürden, welche die Umsetzung leider schlussendlich nicht möglich machten.
\newline
Die wichtigsten Erkenntnissen und Schlussfolgerungen sind:

\section{Technische Herausforderungen}
Die Implementierung der Steuerung eines humanoiden Roboters über eine VR-Brille wie die MetaQuest 3 erfordert eine präzise und latenzfreie Datenübertragung. Probleme mit der Netzwerkverbindung, Synchronisation und Verzögerungen führten dazu, dass die Steuerung von Pepper nicht in der gewünschten Präzision umgesetzt werden konnte. Diese technischen Herausforderungen erwiesen sich als besonders schwerwiegend und konnten trotz intensiver Bemühungen nicht vollständig überwunden werden.

\section{Komplexität der Integration}
Die Integration von Unity und ROS über den ROS-TCP-Connector stellte sich als komplexer als erwartet heraus. Obwohl Unity eine benutzerfreundliche Entwicklungsumgebung bietet, waren die Anforderungen an die ROS-Integration höher als erwartet, insbesondere im Hinblick auf Echtzeitfähigkeit und die nahtlose Zusammenarbeit der verschiedenen Softwarekomponenten. Die Komplexität dieser Integration erwies sich als Hindernis für die erfolgreiche Umsetzung des Projekts.

\section{Wahl von Unity}
Die Entscheidung für Unity als Entwicklungsplattform wurde aufgrund ihrer weiten Verbreitung, der umfangreichen Dokumentation und der großen Entwicklergemeinschaft getroffen. Unity bietet viele vorgefertigte Lösungen und Plugins, die die Entwicklung beschleunigen können. Dennoch stellte sich heraus, dass für diese spezielle Aufgabe tiefergehende Kenntnisse im Umgang mit der Unity-ROS-Integration erforderlich waren, als in der gegebenen Zeit erlernbar war. Diese Lernerfahrung zeigt deutlich die Bedeutung einer sorgfältigen Evaluierung und Auswahl der Entwicklungsplattform für komplexe Projekte.

\section{Potenzial für zukünftige Arbeiten}
Trotz der Herausforderungen zeigt das Projekt deutlich das Potenzial, das in der Kombination von VR und humanoider Robotik liegt. Künftige Arbeiten könnten von den gemachten Erfahrungen profitieren und die entwickelten Ansätze weiter verfolgen. Dies könnte die Erforschung alternativer Technologien für die VR-Steuerung, die Optimierung der Netzwerkverbindung und die Weiterentwicklung der Unity-ROS-Integration umfassen. Welche Schritte zur Weiterführung notwendig sind, werden auch im nachfolgenden Kapitel genauer erklärt.
\\

\noindent
Insgesamt war das Projekt eine wertvolle Lernerfahrung, die Einblicke in die komplexe Welt der Robotik und Virtual Reality ermöglichte. Die gesammelten Erfahrungen und Erkenntnisse werden als Grundlage für zukünftige Entwicklungen dienen, und wir sind zuversichtlich, dass die Vision einer nahtlosen Teleoperation von humanoiden Robotern über VR-Systeme mit weiterem Engagement und Forschung realisiert werden kann. Trotz des Fehlschlags des aktuellen Projekts sind wir motiviert, die gewonnenen Erkenntnisse in zukünftige Projekte zu integrieren und unser Wissen weiter auszubauen.
