\chapter{Fortsetzung des Projekts}
Um das Projekt \textit{Pepper VR – Teleoperation eines humanoiden Roboters auf Basis der Analyse menschlicher Bewegung} in Zukunft erfolgreich umzusetzen, sind mehrere wichtige Schritte und Verbesserungen erforderlich. Nach der Analyse der bisherigen Herausforderungen und der gewonnenen Erkenntnisse haben wir folgende Empfehlungen für die Fortsetzung des Projekts zusammengestellt:

\section{Optimierung der technischen Infrastruktur}
Eine der größten Herausforderungen war die präzise und latenzfreie Datenübertragung zwischen der MetaQuest 3 VR-Brille und Pepper. Zur Verbesserung der technischen Infrastruktur sollten folgende Maßnahmen ergriffen werden:
\begin{itemize}
    \item \textbf{Verbesserung der Netzwerkverbindung}: Sicherstellen einer stabilen und schnellen Netzwerkverbindung, möglicherweise durch den Einsatz von dedizierten Netzwerken oder der Optimierung der bestehenden Infrastruktur.
    \item \textbf{Reduzierung der Latenzzeiten}: Implementierung von Echtzeit-Optimierungen sowohl auf der Hardware- als auch auf der Softwareseite, um die Verzögerungen bei der Datenübertragung zu minimieren.
\end{itemize}

\section{Erweiterung der Softwareintegration}
Die Integration von Unity und ROS über den ROS-TCP-Connector muss weiter verfeinert und genauer recherchiert werden. Hier sind die wesentlichen Schritte:
\begin{itemize}
    \item \textbf{Tiefere ROS-Integration}: Entwicklung zusätzlicher ROS-Nodes und -Services, um eine nahtlosere Kommunikation zwischen Unity und dem Robot Operating System zu gewährleisten.
    \item \textbf{Fehlersuche und Debugging}: Intensive Fehlersuche und Debugging der bestehenden Implementierung, um die Ursachen für Synchronisationsprobleme zu identifizieren und zu beheben.
\end{itemize}

\section{Testen und Validieren der VR-Steuerung}
Um die Steuerung von Pepper über die MetaQuest 3 VR-Brille erfolgreich umzusetzen, sind umfangreiche Tests und Validierungen notwendig:
\begin{itemize}
    \item \textbf{Erstellung von Testfällen}: Entwicklung spezifischer Testfälle, die die verschiedenen Aspekte der VR-Steuerung abdecken, um sicherzustellen, dass alle Funktionen wie erwartet arbeiten.
    \item \textbf{Benutzerstudien}: Durchführung von Benutzerstudien, um die Benutzerfreundlichkeit und Effektivität der VR-Steuerung zu bewerten und basierend auf dem Feedback Verbesserungen vorzunehmen.
\end{itemize}

\section{Schulung und Dokumentation}
Die erfolgreiche Umsetzung des Projekts erfordert gut geschulte Teammitglieder und umfassende Dokumentation:
\begin{itemize}
    \item \textbf{Schulung der Teammitglieder}: Regelmäßige Schulungen und Workshops für das Team, um sicherzustellen, dass alle Mitglieder die notwendigen Kenntnisse und Fähigkeiten haben, um mit den verwendeten Technologien effektiv zu arbeiten.
    \item \textbf{Erstellung umfassender Dokumentation}: Dokumentation aller Implementierungs- und Testprozesse, um eine klare und nachvollziehbare Basis für zukünftige Arbeiten zu schaffen.
\end{itemize}

\section{Langfristige Wartung und Weiterentwicklung}
Um das Projekt langfristig erfolgreich zu halten, sollten regelmäßige Wartung und Weiterentwicklungen eingeplant werden:
\begin{itemize}
    \item \textbf{Regelmäßige Updates}: Kontinuierliche Aktualisierung der verwendeten Software und Hardware, um mit den neuesten Entwicklungen Schritt zu halten und die Stabilität und Sicherheit zu gewährleisten.
    \item \textbf{Weiterentwicklung der Funktionen}: Fortlaufende Erweiterung der Funktionen und Fähigkeiten des Systems, basierend auf den Bedürfnissen und Rückmeldungen der Benutzer.
\end{itemize}

Durch die Umsetzung dieser Maßnahmen kann das Projekt \textit{Pepper VR} erfolgreich weitergeführt werden, um das volle Potenzial der Kombination von Virtual Reality und humanoider Robotik auszuschöpfen. Die gewonnenen Erkenntnisse und Erfahrungen bilden eine solide Grundlage für zukünftige Entwicklungen und Innovationen in diesem spannenden und zukunftsträchtigen Bereich.
