\chapter{Fortsetzung des Projekts}
Um das Projekt \textit{Pepper VR – Teleoperation eines humanoiden Roboters auf Basis der Analyse menschlicher Bewegung} in Zukunft erfolgreich umzusetzen, sind mehrere wichtige Schritte und Verbesserungen erforderlich. Nach der Analyse der bisherigen Herausforderungen und der gewonnenen Erkenntnisse haben wir folgende Empfehlungen für die Fortsetzung des Projekts zusammengestellt:

\section{Optimierung der technischen Infrastruktur}
Eine der größten Herausforderungen war die präzise und latenzfreie Datenübertragung zwischen der MetaQuest 3 VR-Brille und Pepper. Zur Verbesserung der technischen Infrastruktur sollten folgende Maßnahmen ergriffen werden:
\begin{itemize}
    \item \textbf{Verbesserung der Netzwerkverbindung}: Sicherstellen einer stabilen und schnellen Netzwerkverbindung, möglicherweise durch den Einsatz von dedizierten Netzwerken oder der Optimierung der bestehenden Infrastruktur. Dies kann durch die Verwendung von leistungsfähigeren Routern, Switches und Netzwerkkabeln erreicht werden.
    \item \textbf{Reduzierung der Latenzzeiten}: Implementierung von Echtzeit-Optimierungen sowohl auf der Hardware- als auch auf der Softwareseite, um die Verzögerungen bei der Datenübertragung zu minimieren. Dies kann durch die Optimierung von Netzwerkeinstellungen, die Auswahl von leistungsfähigeren Hardwarekomponenten und die Verwendung von effizienteren Datenübertragungsprotokollen erreicht werden.
\end{itemize}

\section{Erweiterung der Softwareintegration}
Die Integration von Unity und ROS über den ROS-TCP-Connector muss weiter verfeinert und genauer recherchiert werden. Hier sind die wesentlichen Schritte:
\begin{itemize}
    \item \textbf{Tiefere ROS-Integration}: Entwicklung zusätzlicher ROS-Nodes und -Services, um eine nahtlosere Kommunikation zwischen Unity und dem Robot Operating System zu gewährleisten. Dies kann durch die Entwicklung benutzerdefinierter ROS-Pakete und die Integration von ROS-Bibliotheken in Unity erreicht werden.
    \item \textbf{Fehlersuche und Debugging}: Intensive Fehlersuche und Debugging der bestehenden Implementierung, um die Ursachen für Synchronisationsprobleme zu identifizieren und zu beheben. Dies kann durch die Verwendung von Debugging-Tools wie ROS-Bag oder ROS-Log und die systematische Analyse von Protokollen und Nachrichtenverläufen erreicht werden.
\end{itemize}

\section{Testen und Validieren der VR-Steuerung}
Um die Steuerung von Pepper über die MetaQuest 3 VR-Brille erfolgreich umzusetzen, sind umfangreiche Tests und Validierungen notwendig:
\begin{itemize}
    \item \textbf{Erstellung von Testfällen}: Entwicklung spezifischer Testfälle, die die verschiedenen Aspekte der VR-Steuerung abdecken, um sicherzustellen, dass alle Funktionen wie erwartet arbeiten. Dies kann durch die Erstellung von automatisierten Testskripten und die systematische Durchführung von Benutzertests in verschiedenen Szenarien und Umgebungen erreicht werden.
    \item \textbf{Benutzerstudien}: Durchführung von Benutzerstudien, um die Benutzerfreundlichkeit und Effektivität der VR-Steuerung zu bewerten und basierend auf dem Feedback Verbesserungen vorzunehmen. Dies kann durch die Durchführung von Umfragen, Interviews und Beobachtungsstudien mit potenziellen Benutzern in realen Einsatzszenarien erreicht werden.
\end{itemize}
\section{Benutzerstudie zur Qualitätssicherung}

Die in Kapitel X beschriebene Benutzerstudie und Evaluation kann als wesentliches Instrument zur Qualitätssicherung dienen, um das entwickelte VR-basierte Steuerungssystem für den humanoiden Roboter weiter zu verbessern und zu verfeinern. Die folgenden Schritte können dabei helfen eine Erfolgreiche Studie durchzuführen.

\subsection{Ziele der Benutzerstudie}

Die Ziele dieser Benutzerstudie könnten darin bestehen, die Leistungsfähigkeit und Genauigkeit der VR-Steuerung genauer zu überprüfen und bewerten, mögliche Schwachstellen oder Probleme zu identifizieren und Feedback von potenziellen Benutzern zu sammeln, um die Benutzererfahrung zu optimieren.

\subsection{Teilnehmerauswahl}

Für die Teilnahme an der Studie könnten Personen ausgewählt werden, die die Zielgruppe des Systems repräsentieren, wie beispielsweise Schüler, Studenten oder Mitarbeiter in Produktionen oder Pflegeeinrichtungen.

\subsection{Experimentelles Design}

Das experimentelle Design sollte verschiedene Aspekte der VR-Steuerung abdecken, darunter Genauigkeit, Benutzerfreundlichkeit, Immersion und Reaktionszeit. Es sollten Aufgaben entworfen werden, die die Fähigkeiten und Grenzen des Systems in verschiedenen Anwendungsgebieten herausfordern und eine umfassende Bewertung ermöglichen.

\subsection{Durchführung der Studie}

Die Studie sollte sorgfältig geplant und durchgeführt werden, wobei klare Anweisungen für die Teilnehmer bereitgestellt werden und die Interaktionen genau protokolliert werden, um aussagekräftige Daten zu erfassen.

\subsection{Datenerfassung und Analyse}

Quantitative Metriken wie Genauigkeit, Effizienz und Reaktionszeit können gemessen werden, während qualitative Feedbacks durch Interviews, Fragebögen oder offene Diskussionen gesammelt werden, um Einsichten in die Benutzererfahrung zu gewinnen.

\subsection{Interpretation der Ergebnisse}

Die gesammelten Daten sollten gründlich analysiert werden, um Trends, Muster und potenzielle Probleme zu identifizieren. Die Interpretation des Feedbacks der Teilnehmer kann wertvolle Einblicke liefern, um Verbesserungsmöglichkeiten zu erkennen.

\subsection{Schlussfolgerungen und Empfehlungen}

Basierend auf den Ergebnissen der Studie können Schlussfolgerungen gezogen und Empfehlungen abgeleitet werden, um das VR-Steuerungssystem für den humanoiden Roboter weiter zu optimieren und die Benutzererfahrung zu verbessern.

Die Durchführung einer solchen Benutzerstudie zur Qualitätssicherung kann dazu beitragen, das entwickelte System noch besser an die Bedürfnisse und Anforderungen der Benutzer anzupassen und letztendlich die Zufriedenheit und Akzeptanz zu steigern.

\section{Schulung und Dokumentation}
Die erfolgreiche Umsetzung des Projekts erfordert gut geschulte Teammitglieder und umfassende Dokumentation:
\begin{itemize}
    \item \textbf{Schulung der Teammitglieder}: Regelmäßige Schulungen und Workshops für das Team, um sicherzustellen, dass alle Mitglieder die notwendigen Kenntnisse und Fähigkeiten haben, um mit den verwendeten Technologien effektiv zu arbeiten. Dies kann durch die Organisation von Schulungsveranstaltungen, die Bereitstellung von Schulungsmaterialien und die Einrichtung von Online-Lernplattformen erreicht werden.
    \item \textbf{Erstellung umfassender Dokumentation}: Dokumentation aller Implementierungs- und Testprozesse, um eine klare und nachvollziehbare Basis für zukünftige Arbeiten zu schaffen. Dies kann durch die Erstellung von technischen Handbüchern, Benutzerhandbüchern, API-Dokumentationen und Projektberichten erreicht werden.
\end{itemize}

\section{Langfristige Wartung und Weiterentwicklung}
Um das Projekt langfristig erfolgreich zu halten, sollten regelmäßige Wartung und Weiterentwicklungen eingeplant werden:
\begin{itemize}
    \item \textbf{Regelmäßige Updates}: Kontinuierliche Aktualisierung der verwendeten Software und Hardware, um mit den neuesten Entwicklungen Schritt zu halten und die Stabilität und Sicherheit zu gewährleisten. Dies kann durch die regelmäßige Überprüfung und Aktualisierung von Bibliotheken, Frameworks und Betriebssystemen erreicht werden.
    \item \textbf{Weiterentwicklung der Funktionen}: Fortlaufende Erweiterung der Funktionen und Fähigkeiten des Systems, basierend auf den Bedürfnissen und Rückmeldungen der Benutzer. Dies kann durch die Implementierung neuer Features, die Integration von externen APIs und die Entwicklung von Erweiterungsmodulen erreicht werden.
\end{itemize}

\noindent
Durch die Umsetzung dieser Maßnahmen kann das Projekt \textit{Pepper VR} erfolgreich weitergeführt werden, um das volle Potenzial der Kombination von Virtual Reality und humanoider Robotik auszuschöpfen. Die gewonnenen Erkenntnisse und Erfahrungen bilden eine solide Grundlage für zukünftige Entwicklungen und Innovationen in diesem spannenden und zukunftsträchtigen Bereich.
