\chapter{Technologieauswahl}
\section{Pepper}
Die Auswahl des humanoiden Roboters Pepper für das Projekt, wurde aufgrund seiner vielseitigen Anwendungsmöglichkeiten und seiner Benutzerfreundlichkeit getroffen. Die Hauptgründe für die Auswahl von Pepper sind im Folgenden erläutert:
\begin{enumerate}
    \item \ac{HRI} Entwickelt für menschliche Interaktion, ist Pepper für die Interaktion mit Menschen in einem sozialen Kontext konzipiert. Sein humanoides Aussehen, seine ausdrucksstarken Gesten und seine Sprachfähigkeiten machen ihn besonders geeignet für die Interaktion mit Menschen und um ihn in sozialen Umgebungen einzusetzen. Für seine Spracherkennung und Sprachsynthese ist er ausgestattet mit fortschrittlicher Spracherkennung und -synthese, womit er verbale Befehle verstehen und darauf reagieren und sich in mehreren Sprachen unterhalten kann. Diese Funktionalität wird erweitert mit benutzerdefinierten Sprachbefehlen und Dialogen, die es ermöglichen, Pepper an spezifische Anforderungen anzupassen und ihn in auch inf spezielleren Anwendungsfällen einzusetzen.
    \item Bei seiner Mobilität und autonomen Navigation verfügt Pepper über integrierte und autonome Navigationsfähigkeiten, die es ihm ermöglichen, sich in dynamischen Umgebungen zu bewegen. Er kann Hindernissen ausweichen und zu bestimmten Orten innerhalb eines kartierten Bereichs navigieren. Dies wird unter anderem ermöglicht durch seine Omnidirektionale Räder im Fahrwerk. Peppers Basis ist mit omnidirektionalen Rädern ausgestattet, die sanfte und präzise Bewegungen in alle Richtungen ermöglichen und seine Fähigkeit zur Interaktion auf engem Raum verbessern. Drehen auf der Stelle und seitwärts fahren sind durch diese Technologie ebenfalls möglich und gerade in dynamischen und sozialen Umgebungen von Vorteil.
    \item Auch bei der Sensorik ist Pepper voll ausgestattet. Er verfügt über eine Reihe von Sensoren, darunter Berührungssensoren, Sonare, Laser und Tiefenkameras, die es ihm ermöglichen, seine Umgebung effektiv wahrzunehmen und mit ihr zu interagieren. Diese Sensorikausstattung ist soweit ausgereift, dass Pepper menschliche Gesichter erkennen kann und Personen verfolgen und Emotionen erkennen kann, was für Anwendungen, die menschliche Interaktion und Engagement erfordern, nützlich ist.
    \item Für Entwickler bietet der Hersteller SoftBank das NAOqi-Framework an. Pepper verwendet das NAOqi-Framework, das eine Vielzahl von \ac{API}s für die Entwicklung von Anwendungen bietet. Dieses Framework ist multilingual und unterstützt die Sprachen C++, Python, Java und sogar für \ac{ROS} existieren Schnittstellen, womit für eine breite Palette gesorgt wird und vielen von Entwicklern das entwickeln zugänglich gemacht wird. Besonders ist hierbei die \ac{ROS}-Kompatibilität, durch die Pepper in größere Robotersysteme integriert werden und von den umfangreichen Bibliotheken und Werkzeugen profitieren, die im \ac{ROS}-Ökosystem verfügbar sind. Diese \ac{ROS} Kompatibilität ist auch in diesem Projekt eine der Hauptgründe für die Auswahl von Pepper.
    \item Typischerweise wird Pepper in Positionen verwendet, in denen er mit Menschen interagiert, wie beispielsweise in der Bildung, im Einzelhandel, in der Unterhaltungsbranche oder an Theken. Seine Fähigkeit, mit Menschen zu interagieren und sie zu unterhalten, macht ihn zu einem idealen Kandidaten für das Projekt, bei dem er als Schnittstelle zwischen dem Benutzer und der \ac{VR}-Umgebung dient. Durch die Integration von Pepper in das \ac{VR}-Erlebnis wird es möglich, eine immersive und interaktive Erfahrung zu schaffen, bei der die Benutzer direkt mit dem Roboter interagieren können.
    \item Ein weiterer großer Vorteil von Pepper ist seine Anpassbarkeit und Erweiterbarkeit. Der Hersteller und die Community unterstützen aktiv das App-Ökosystem. Dadurch können Entwickler mithilfe der von SoftBank Robotics bereitgestellten \ac{SDK}s benutzerdefinierte Anwendungen für Pepper erstellen. Dies ermöglicht maßgeschneiderte Lösungen für spezifische Probleme oder Anwendungsfälle. Schon die Basis der Software-Architektur des Roboters ermöglicht das Hinzufügen von benutzerdefinierten Modulen und Funktionalitäten, so dass Entwickler die Fähigkeiten von Pepper nach Bedarf erweitern können.
    \item Zur Interaktion mit Pepper hat er auf der Brust einen Touchscreen, der zur Anzeige von Informationen, Bildern und Videos verwendet werden kann und auch als Eingabegerät für den Benutzer dient. Weiter kann Pepper sich auch über seine Ausdrucksstarke Animationen ausdrücken. Mit seiner Fähigkeit, verschiedene Animationen und Gesten auszuführen, kann Pepper Emotionen ausdrücken und dem Benutzer eine ansprechende Erfahrung bieten.
    \item Schon seit einigen Jahren ist Pepper auf dem Markt und hat sich in verschiedenen Branchen etabliert. Dadurch gibt es nicht nur die Akzeptanz eines Roboters in verschiedenen Bereichen, sondern auch eine große Community von Entwicklern und Unternehmen, die Anwendungen und Lösungen für Pepper entwickeln. Und auch der Hersteller sebst bietet umfangreiche Unterstützung, Ressourcen und Dokumentationen an, um mit Pepper zu entwickeln und ihn in verschiedenen Anwendungen einzusetzen.
\end{enumerate}
Allgemein ist das Design und die Fähigkeiten von Pepper vielseitig und freundlich und machen ihn so zu einer vielseitigen und effektiven Plattform für Anwendungen, die eine \ac{HRI} erfordern. Sein umfangreiches Sensorpaket, seine autonome Navigation und sein Entwicklungsrahmen in Kombination mit seinem ausdrucksstarken und ansprechenden Design machen ihn zu einem starken Konkurrenten im Bereich der sozialen Robotik, und für dieses Projekt die Ideale Wahl.

\subsection{Alternativen}
Auch wenn Pepper in diesem Projekt gewählt wurde, ist er auf dem Markt der humanoiden Roboter nicht alleine und bekommt sogar Konkurenz aus dem eigenen Haus. Ein paar nennenswerte Alternativen wären gewesen:
\begin{itemize}
    \item Der im eigenen Haus entwickelte NAO Roboter. Auch er ist humanoid und besitzt Sprachrekennung und Emotionen. Allerdings ist er weitaus kleiner und daher in diesem Anwendungsfall nicht present genug, wodurch er in belebteren Umgebungen leicht untergeh und beschädigt werden könnte. Er basiert auf dem selben Betriebsystem NAOqi, welchem er auch den Namen verdankt.
    \item Der Furhat-Robots legen mit ihren Produkten den Fokus primär auf Ausdrücke und Emotionen. In diesem Feld sind sie Pepper auch weit überlegen. Allerdings besitzen diese Roboter keinen Torso und keine Möglickeit sich fortzubewegen, was sie in diesem Fall auch nicht als nützlich erweist.
    \item Der Misty-2 der gleichnamigen Firma Misty-Robotics ist ebenfalls entwickelt für zwischenmenschliche Interaktionen. Jedoch ist er ebenfalls sehr klein und besitzt zwar einen Torso und Arme, kann mit diesen aber nichts greifen. Daher scheidet er auch aus dem Rennen aus
\end{itemize}
Letztlich muss man jedoch auch erwähnen, dass für diese Projekt primär der Pepper Roboter vorgeschlagen wurde und die Wahl auch durch die gegebenen Umstände und Materialien beeinflusst wurde.

\section{ROS}
Einer der Hauptvorteile von \ac{ROS} ist seine Modularität und Wiederverwendbarkeit. Die modulare Architektur ermöglicht es den Entwicklern, die Robotersoftware in kleine, wiederverwendbare Komponenten, so genannte Nodes, zu zerlegen, die jeweils bestimmte Aufgaben übernehmen. Dies fördert nicht nur eine bessere Organisation und Wartung, sondern ermöglicht auch eine erhebliche Wiederverwendung von Code. Die große Anzahl an Paketen und Bibliotheken, die in ROS zur Verfügung stehen, beschleunigt die Entwicklung, da die Entwickler vorhandene Funktionen wiederverwenden und integrieren können.\\
Die sogenannte Interoperabilität ist ein weiterer wichtiger Vorteil von \ac{ROS}. Es standardisiert die Kommunikation zwischen Knoten durch das Publish-/Subscribe-Nachrichtenmodell, das die nahtlose Integration verschiedener Softwarekomponenten und Sensoren ermöglicht, unabhängig von der zugrunde liegenden Implementierung. Darüber hinaus unterstützt \ac{ROS} mehrere Programmiersprachen, hauptsächlich Python und C++ so, dass Entwickler die Sprache wählen können, die ihren Aufgaben am besten entspricht.\\
Auch das umfangreiche Ökosystem von \ac{ROS} ist ein entscheidender Vorteil. Mit einer großen und aktiven Community von Entwicklern und Forschern bietet \ac{ROS} umfangreiche Unterstützung durch Foren, Tutorials und Dokumentation. Die große Auswahl an verfügbaren Paketen für verschiedene Roboterfunktionen wie Wahrnehmung, Navigation, Manipulation und Steuerung erleichtert das schnelle Prototyping und die Entwicklung.\\
\ac{ROS} bietet leistungsstarke Tools für Debugging, Visualisierung und Simulation. Mit Tools wie \textit{Rviz} und \textit{rqt} können Entwickler Sensordaten und Roboterzustände visualisieren beziehungsweise grafische Benutzeroberflächen erstellen. Weiter lässt sich \c{ROS} gut in Simulationsumgebungen wie etwa \textit{Gazebo} integrieren, so dass Entwickler ihre Algorithmen in einer simulierten Welt testen und validieren können, bevor sie sie auf physischen Robotern einsetzen.\\
Auch im Thema Skalierbarkeit bietet \ac{ROS} umfangreiche Unterstüzung. Es wurde für die Unterstützung verteilter Datenverarbeitung entwickelt und ermöglicht die Zusammenarbeit mehrerer Computer und Roboter in vernetzten Umgebungen, was für komplexe Robotersysteme, die erhebliche Rechenressourcen benötigen, unerlässlich ist. Seine Architektur macht es einfach, neue Funktionen hinzuzufügen und bestehende zu erweitern, so dass sich das System mit den sich ändernden Anforderungen weiterentwickeln kann und Komponenten unabhängig bleiben.\\
\ac{ROS} ist sowohl in der Wissenschaft als auch in der Industrie weit verbreitet und hat sich zu einem Standard in der Robotikgemeinschaft entwickelt. Seine Verwendung in Spitzenforschungsprojekten stellt sicher, dass es bei der Weiterentwicklung der Robotertechnologie an vorderster Front bleibt. Die akademische und industrielle Akzeptanz von ROS unterstreicht seine Zuverlässigkeit und Effektivität in verschiedenen Anwendungen.\\
Ein weiterer Hauptgrund für die Wahl von \ac{ROS} ist, dass \ac{ROS} ein Open-Source-Projekt ist. Das ermöglicht auch Einzelpersonen, Start-ups und Organisationen kostengünstig und ohne teure Lizenzgebühren eine mächtige Roboterlösung. Der Open-Source-Charakter von \ac{ROS} ermöglicht es Entwicklern auch, den Quellcode zu prüfen, zu ändern und an spezielle Anforderungen anzupassen, wodurch die volle Kontrolle über die Software gewährleistet ist und \ac{ROS} immer auf dem aktuellsten Stand der Technik und Sicherheit hält.\\
Abschließend ist, dass \ac{ROS} ein robustes, flexibles und gut unterstütztes Framework für die Entwicklung von Roboteranwendungen. Seine modulare Architektur, das umfangreiche Ökosystem, die leistungsstarken Tools und die aktive Community machen es zu einer ausgezeichneten Wahl für dieses Projekt.\\




\section{VR-Brillen}
Die Auswahl der Technologie für die Entwicklung von Virtual-Reality-Anwendungen ist von entscheidender Bedeutung für den Erfolg eines Projekts. Bei der DHBW wurde die Entscheidung getroffen, die Metaquest 3 für das VR-Projekt zu verwenden. Diese Entscheidung wurde aufgrund mehrerer Faktoren getroffen, die im Folgenden erläutert werden.
\\

\noindent
Die Metaquest 3 wurde aufgrund ihrer vielseitigen Anwendungsmöglichkeiten und ihrer Benutzerfreundlichkeit ausgewählt. Die DHBW legt großen Wert darauf, den Studierenden eine moderne und zugängliche Lernumgebung zu bieten. Die Metaquest 3 erfüllt diese Anforderungen durch ihre intuitive Bedienung und ihre Fähigkeit, komplexe VR-Erlebnisse bereitzustellen, ohne dass zusätzliche Hardware wie externe Sensoren erforderlich sind.
\\

\noindent
Ein weiterer wichtiger Faktor bei der Auswahl der Metaquest 3 war die Integration von Oculus in die bestehende IT-Infrastruktur der DHBW. Die Unterstützung und Zusammenarbeit mit Oculus ermöglichte es der DHBW, Schulungen und Support für die Verwendung der Metaquest 3 bereitzustellen. Dies erleichterte die Einführung der VR-Technologie in den Lehrplan und sorgte für eine nahtlose Integration in bestehende Lehr- und Lernaktivitäten.
\\

\noindent
Darüber hinaus bietet die Metaquest 3 eine breite Palette von Anwendungen und Inhalten über den Oculus Store, einschließlich Bildungs- und Trainingsanwendungen, die für den Einsatz in der Hochschulbildung geeignet sind. Die Verfügbarkeit von hochwertigen Bildungsressourcen spielte eine wichtige Rolle bei der Entscheidung für die Metaquest 3, da sie den Lehrern und Studierenden Zugang zu einer Vielzahl von Lernmaterialien und Simulationen bietet, die den Lernprozess unterstützen und verbessern können.
\\

\noindent
Insgesamt wurde die Metaquest 3 aufgrund ihrer Benutzerfreundlichkeit, Integration in die bestehende Infrastruktur der DHBW und der Verfügbarkeit von Bildungsressourcen als ideale Wahl für das VR-Projekt der DHBW angesehen. Diese Auswahl bietet nicht nur eine solide Grundlage für die Entwicklung von VR-Anwendungen, sondern ermöglicht es auch, die VR-Technologie effektiv in den Lehrplan zu integrieren und den Lernerfolg zu steigern.
\section{Entwicklung für VR-Brillen}
Die Entscheidung, Unity als Entwicklungsplattform für das VR-Projekt "Pepper VR – Teleoperation eines humanoiden Roboters auf Basis der Analyse menschlicher Bewegung" zu wählen, wurde nach sorgfältiger Abwägung verschiedener Faktoren getroffen, wobei insbesondere auch die Unreal Engine in Betracht gezogen wurde.

\subsubsection{Unity}
Im folgenden werden die Vor- und Nachteile von Unity beleuchtet.
\paragraph{Vorteile:}

\begin{enumerate}
\item \textbf{Branchenübliche Plattform}: Unity ist eine der führenden Entwicklungsplattformen für VR-Anwendungen und wird von einer großen Community von Entwicklern und Unternehmen weltweit genutzt. Diese weitverbreitete Akzeptanz macht Unity zu einer branchenüblichen Wahl für die Entwicklung von VR-Inhalten und bietet Zugang zu einer Vielzahl von Ressourcen, Tutorials und Support, die für die erfolgreiche Umsetzung des Projekts entscheidend sind.
\item \textbf{Umfangreiche Funktionalitäten}: Unity bietet eine umfangreiche Auswahl an Funktionen und Werkzeugen, die speziell für die Entwicklung von VR-Anwendungen konzipiert sind. Die Integration von VR-Technologien wie Oculus Rift, HTC Vive und Metaquest in Unity ermöglicht es den Entwicklern, immersive VR-Erlebnisse mit hoher Qualität zu erstellen. Darüber hinaus bietet Unity eine benutzerfreundliche Oberfläche und eine intuitive Entwicklungsumgebung, die auch für Anfänger leicht zugänglich ist.

\item \textbf{Plattformübergreifende Unterstützung}: Unity ermöglicht die Entwicklung von VR-Anwendungen, die auf einer Vielzahl von Plattformen ausgeführt werden können, einschließlich PC, Konsolen, Mobilgeräten und Webbrowsern. Diese Flexibilität eröffnet die Möglichkeit, das VR-Projekt auf verschiedenen Geräten und Betriebssystemen zu testen und bereitzustellen, um eine maximale Reichweite und Zugänglichkeit zu gewährleisten.

\item \textbf{Erweiterbarkeit und Anpassbarkeit}: Unity zeichnet sich durch seine Erweiterbarkeit und Anpassbarkeit aus. Durch den Einsatz von Plugins und Assets aus dem Unity Asset Store können Entwickler zusätzliche Funktionen und Ressourcen in ihre VR-Anwendungen integrieren, was die Entwicklung beschleunigt und die Produktivität erhöht. Darüber hinaus bietet Unity die Möglichkeit, eigene Tools und Skripte zu erstellen, um die spezifischen Anforderungen des Projekts zu erfüllen und maßgeschneiderte Lösungen zu entwickeln.
\end{enumerate}

\paragraph{Nachteile:}

\begin{enumerate}
\item \textbf{Grafische Qualität}: Obwohl Unity in Bezug auf die Grafikqualität fortschrittliche Techniken bietet, erreicht es möglicherweise nicht das gleiche grafische Niveau wie die Unreal Engine, insbesondere bei fotorealistischen Rendering-Anforderungen.
\item \textbf{Lernkurve}: Unity kann eine steilere Lernkurve haben als die Unreal Engine, insbesondere für Anfänger oder Personen ohne Programmiererfahrung. Die Vielzahl von Funktionen und Optionen kann anfangs überwältigend sein.
\end{enumerate}

\subsubsection{Unreal Engine}
Nun werden die Vor- und Nachteile von Unreal Engine betrachtet.
\paragraph{Vorteile:}

\begin{enumerate}
\item \textbf{Grafische Qualität}: Die Unreal Engine ist bekannt für ihre beeindruckende Grafikqualität und ihre Fähigkeit, fotorealistische Umgebungen zu erstellen. Sie bietet fortschrittliche Rendering-Techniken wie Raytracing und hochwertige Materialien, die für VR-Anwendungen mit hohen grafischen Anforderungen von Vorteil sind.
\item \textbf{Visuelle Skripting-Tools}: Die Unreal Engine bietet visuelle Skripting-Tools wie den Blueprint-Editor, die es auch Personen ohne umfangreiche Programmierkenntnisse ermöglichen, komplexe Logik und Interaktionen zu erstellen. Dies kann die Entwicklungszeit verkürzen und die Kreativität fördern.

\item \textbf{Leistung}: Die Unreal Engine ist für ihre hohe Leistung und Stabilität bekannt, insbesondere bei großen Projekten mit komplexen Szenen und großen Datenmengen.
\end{enumerate}

\paragraph{Nachteile:}

\begin{enumerate}
\item \textbf{Einschränktere Plattformunterstützung}: Im Vergleich zu Unity bietet die Unreal Engine möglicherweise eine eingeschränktere Plattformunterstützung für die Entwicklung von VR-Anwendungen. Die Unterstützung für bestimmte VR-Geräte oder Plattformen kann begrenzt sein.
\item \textbf{Komplexität}: Die Unreal Engine kann aufgrund ihrer fortschrittlichen Funktionen und der visuellen Komplexität ihrer Benutzeroberfläche für Anfänger schwieriger zu erlernen sein. Die Blueprint-Logik kann zwar visuell sein, erfordert jedoch immer noch ein Verständnis von Konzepten wie Variablen und Logik.
\end{enumerate}

\subsubsection{Entscheidung}

Trotz der Vorteile der Unreal Engine in Bezug auf Grafikqualität und Leistung wurde Unity als die bevorzugte Entwicklungsplattform für das VR-Projekt "Pepper VR – Teleoperation eines humanoiden Roboters auf Basis der Analyse menschlicher Bewegung" gewählt. Die breite Unterstützung, die umfangreichen Funktionalitäten, die plattformübergreifende Unterstützung und die Erweiterbarkeit von Unity waren entscheidend für diese Wahl. Unity bietet eine solide Grundlage für die Entwicklung hochwertiger VR-Anwendungen, die den Anforderungen des Projekts gerecht werden und eine erfolgreiche Integration von VR-Technologie in den Lehrplan ermöglichen.
\subsection{Verbindungstechnologie}
Die Wahl von Unity als Entwicklungsplattform für das VR-Projekt "Pepper VR – Teleoperation eines humanoiden Roboters auf Basis der Analyse menschlicher Bewegung" führte zur Notwendigkeit, eine Methode zur Kommunikation zwischen der Unity-Anwendung und dem humanoiden Roboter Pepper zu implementieren. Die ROS TCP-Verbindung wurde aufgrund ihrer Kompatibilität mit Unity und der Robotersteuerung über das Robot Operating System (ROS) als geeignete Lösung identifiziert.

\subsubsection{Kompatibilität mit Unity}

ROS TCP (Transmission Control Protocol) bietet eine zuverlässige Methode zur Datenübertragung zwischen ROS und Unity. Unity unterstützt die Kommunikation über TCP/IP-Sockets, was es ermöglicht, Daten zwischen der Unity-Anwendung und externen Geräten wie Robotern über das Netzwerk auszutauschen. Durch die Implementierung einer ROS TCP-Verbindung kann die Unity-Anwendung Befehle an den Roboter senden und Daten von ihm empfangen, was eine nahtlose Integration in das VR-Erlebnis ermöglicht.

\subsubsection{Robotersteuerung über ROS}

Pepper, der humanoide Roboter, wird über das Robot Operating System (ROS) gesteuert, das eine Standardplattform für die Entwicklung von Robotersoftware ist. ROS bietet eine Vielzahl von Funktionen zur Robotersteuerung, einschließlich der Unterstützung für verschiedene Sensoren, Aktuatoren und Navigationssysteme. Indem die ROS TCP-Verbindung verwendet wird, kann die Unity-Anwendung mit den ROS-Nodes kommunizieren, die für die Steuerung von Pepper zuständig sind. Dies ermöglicht es der Unity-Anwendung, Pepper-Bewegungen zu steuern und Sensordaten von Pepper zu empfangen, um ein interaktives VR-Erlebnis zu schaffen.

\subsubsection{Implementierung in Unity}

Die Implementierung der ROS TCP-Verbindung in Unity erfolgt mithilfe von Plugins oder eigenen Skripten, die die TCP/IP-Kommunikation ermöglichen. Durch die Verwendung von vorhandenen ROS-Bibliotheken oder der Erstellung benutzerdefinierter ROS-Nodes kann die Unity-Anwendung ROS-Nachrichten senden und empfangen, um mit Pepper zu interagieren. Dies ermöglicht es, die Bewegungen von Pepper in Echtzeit zu steuern und Feedbackdaten von Pepper in die Unity-Anwendung zu integrieren.
\\

\noindent
Insgesamt wurde die ROS TCP-Verbindung aufgrund ihrer Kompatibilität mit Unity und der Robotersteuerung über ROS als ideale Lösung für die Kommunikation zwischen der Unity-Anwendung und dem humanoiden Roboter Pepper ausgewählt. Diese Entscheidung ermöglicht es, eine immersive und interaktive VR-Erfahrung zu schaffen, bei der die Benutzer direkt mit dem Roboter interagieren können.