\chapter{Technologieauswahl}
\section{VR-Brillen}
Die Auswahl der Technologie für die Entwicklung von Virtual-Reality-Anwendungen ist von entscheidender Bedeutung für den Erfolg eines Projekts. Bei der DHBW wurde die Entscheidung getroffen, die Metaquest 3 für das VR-Projekt zu verwenden. Diese Entscheidung wurde aufgrund mehrerer Faktoren getroffen, die im Folgenden erläutert werden.
\\

\noindent
Die Metaquest 3 wurde aufgrund ihrer vielseitigen Anwendungsmöglichkeiten und ihrer Benutzerfreundlichkeit ausgewählt. Die DHBW legt großen Wert darauf, den Studierenden eine moderne und zugängliche Lernumgebung zu bieten. Die Metaquest 3 erfüllt diese Anforderungen durch ihre intuitive Bedienung und ihre Fähigkeit, komplexe VR-Erlebnisse bereitzustellen, ohne dass zusätzliche Hardware wie externe Sensoren erforderlich sind.
\\

\noindent
Ein weiterer wichtiger Faktor bei der Auswahl der Metaquest 3 war die Integration von Oculus in die bestehende IT-Infrastruktur der DHBW. Die Unterstützung und Zusammenarbeit mit Oculus ermöglichte es der DHBW, Schulungen und Support für die Verwendung der Metaquest 3 bereitzustellen. Dies erleichterte die Einführung der VR-Technologie in den Lehrplan und sorgte für eine nahtlose Integration in bestehende Lehr- und Lernaktivitäten.
\\

\noindent
Darüber hinaus bietet die Metaquest 3 eine breite Palette von Anwendungen und Inhalten über den Oculus Store, einschließlich Bildungs- und Trainingsanwendungen, die für den Einsatz in der Hochschulbildung geeignet sind. Die Verfügbarkeit von hochwertigen Bildungsressourcen spielte eine wichtige Rolle bei der Entscheidung für die Metaquest 3, da sie den Lehrern und Studierenden Zugang zu einer Vielzahl von Lernmaterialien und Simulationen bietet, die den Lernprozess unterstützen und verbessern können.
\\

\noindent
Insgesamt wurde die Metaquest 3 aufgrund ihrer Benutzerfreundlichkeit, Integration in die bestehende Infrastruktur der DHBW und der Verfügbarkeit von Bildungsressourcen als ideale Wahl für das VR-Projekt der DHBW angesehen. Diese Auswahl bietet nicht nur eine solide Grundlage für die Entwicklung von VR-Anwendungen, sondern ermöglicht es auch, die VR-Technologie effektiv in den Lehrplan zu integrieren und den Lernerfolg zu steigern.
\section{Entwicklung für VR-Brillen}
Die Entscheidung, Unity als Entwicklungsplattform für das VR-Projekt "Pepper VR – Teleoperation eines humanoiden Roboters auf Basis der Analyse menschlicher Bewegung" zu wählen, wurde nach sorgfältiger Abwägung verschiedener Faktoren getroffen, wobei insbesondere auch die Unreal Engine in Betracht gezogen wurde.

\subsubsection{Unity}
Im folgenden werden die Vor- und Nachteile von Unity beleuchtet.
\paragraph{Vorteile:}

\begin{enumerate}
\item \textbf{Branchenübliche Plattform}: Unity ist eine der führenden Entwicklungsplattformen für VR-Anwendungen und wird von einer großen Community von Entwicklern und Unternehmen weltweit genutzt. Diese weitverbreitete Akzeptanz macht Unity zu einer branchenüblichen Wahl für die Entwicklung von VR-Inhalten und bietet Zugang zu einer Vielzahl von Ressourcen, Tutorials und Support, die für die erfolgreiche Umsetzung des Projekts entscheidend sind.
\item \textbf{Umfangreiche Funktionalitäten}: Unity bietet eine umfangreiche Auswahl an Funktionen und Werkzeugen, die speziell für die Entwicklung von VR-Anwendungen konzipiert sind. Die Integration von VR-Technologien wie Oculus Rift, HTC Vive und Metaquest in Unity ermöglicht es den Entwicklern, immersive VR-Erlebnisse mit hoher Qualität zu erstellen. Darüber hinaus bietet Unity eine benutzerfreundliche Oberfläche und eine intuitive Entwicklungsumgebung, die auch für Anfänger leicht zugänglich ist.

\item \textbf{Plattformübergreifende Unterstützung}: Unity ermöglicht die Entwicklung von VR-Anwendungen, die auf einer Vielzahl von Plattformen ausgeführt werden können, einschließlich PC, Konsolen, Mobilgeräten und Webbrowsern. Diese Flexibilität eröffnet die Möglichkeit, das VR-Projekt auf verschiedenen Geräten und Betriebssystemen zu testen und bereitzustellen, um eine maximale Reichweite und Zugänglichkeit zu gewährleisten.

\item \textbf{Erweiterbarkeit und Anpassbarkeit}: Unity zeichnet sich durch seine Erweiterbarkeit und Anpassbarkeit aus. Durch den Einsatz von Plugins und Assets aus dem Unity Asset Store können Entwickler zusätzliche Funktionen und Ressourcen in ihre VR-Anwendungen integrieren, was die Entwicklung beschleunigt und die Produktivität erhöht. Darüber hinaus bietet Unity die Möglichkeit, eigene Tools und Skripte zu erstellen, um die spezifischen Anforderungen des Projekts zu erfüllen und maßgeschneiderte Lösungen zu entwickeln.
\end{enumerate}

\paragraph{Nachteile:}

\begin{enumerate}
\item \textbf{Grafische Qualität}: Obwohl Unity in Bezug auf die Grafikqualität fortschrittliche Techniken bietet, erreicht es möglicherweise nicht das gleiche grafische Niveau wie die Unreal Engine, insbesondere bei fotorealistischen Rendering-Anforderungen.
\item \textbf{Lernkurve}: Unity kann eine steilere Lernkurve haben als die Unreal Engine, insbesondere für Anfänger oder Personen ohne Programmiererfahrung. Die Vielzahl von Funktionen und Optionen kann anfangs überwältigend sein.
\end{enumerate}

\subsubsection{Unreal Engine}
Nun werden die Vor- und Nachteile von Unreal Engine betrachtet.
\paragraph{Vorteile:}

\begin{enumerate}
\item \textbf{Grafische Qualität}: Die Unreal Engine ist bekannt für ihre beeindruckende Grafikqualität und ihre Fähigkeit, fotorealistische Umgebungen zu erstellen. Sie bietet fortschrittliche Rendering-Techniken wie Raytracing und hochwertige Materialien, die für VR-Anwendungen mit hohen grafischen Anforderungen von Vorteil sind.
\item \textbf{Visuelle Skripting-Tools}: Die Unreal Engine bietet visuelle Skripting-Tools wie den Blueprint-Editor, die es auch Personen ohne umfangreiche Programmierkenntnisse ermöglichen, komplexe Logik und Interaktionen zu erstellen. Dies kann die Entwicklungszeit verkürzen und die Kreativität fördern.

\item \textbf{Leistung}: Die Unreal Engine ist für ihre hohe Leistung und Stabilität bekannt, insbesondere bei großen Projekten mit komplexen Szenen und großen Datenmengen.
\end{enumerate}

\paragraph{Nachteile:}

\begin{enumerate}
\item \textbf{Einschränktere Plattformunterstützung}: Im Vergleich zu Unity bietet die Unreal Engine möglicherweise eine eingeschränktere Plattformunterstützung für die Entwicklung von VR-Anwendungen. Die Unterstützung für bestimmte VR-Geräte oder Plattformen kann begrenzt sein.
\item \textbf{Komplexität}: Die Unreal Engine kann aufgrund ihrer fortschrittlichen Funktionen und der visuellen Komplexität ihrer Benutzeroberfläche für Anfänger schwieriger zu erlernen sein. Die Blueprint-Logik kann zwar visuell sein, erfordert jedoch immer noch ein Verständnis von Konzepten wie Variablen und Logik.
\end{enumerate}

\subsubsection{Entscheidung}

Trotz der Vorteile der Unreal Engine in Bezug auf Grafikqualität und Leistung wurde Unity als die bevorzugte Entwicklungsplattform für das VR-Projekt "Pepper VR – Teleoperation eines humanoiden Roboters auf Basis der Analyse menschlicher Bewegung" gewählt. Die breite Unterstützung, die umfangreichen Funktionalitäten, die plattformübergreifende Unterstützung und die Erweiterbarkeit von Unity waren entscheidend für diese Wahl. Unity bietet eine solide Grundlage für die Entwicklung hochwertiger VR-Anwendungen, die den Anforderungen des Projekts gerecht werden und eine erfolgreiche Integration von VR-Technologie in den Lehrplan ermöglichen.
\subsection{Verbindungstechnologie}
Die Wahl von Unity als Entwicklungsplattform für das VR-Projekt "Pepper VR – Teleoperation eines humanoiden Roboters auf Basis der Analyse menschlicher Bewegung" führte zur Notwendigkeit, eine Methode zur Kommunikation zwischen der Unity-Anwendung und dem humanoiden Roboter Pepper zu implementieren. Die ROS TCP-Verbindung wurde aufgrund ihrer Kompatibilität mit Unity und der Robotersteuerung über das Robot Operating System (ROS) als geeignete Lösung identifiziert.

\subsubsection{Kompatibilität mit Unity}

ROS TCP (Transmission Control Protocol) bietet eine zuverlässige Methode zur Datenübertragung zwischen ROS und Unity. Unity unterstützt die Kommunikation über TCP/IP-Sockets, was es ermöglicht, Daten zwischen der Unity-Anwendung und externen Geräten wie Robotern über das Netzwerk auszutauschen. Durch die Implementierung einer ROS TCP-Verbindung kann die Unity-Anwendung Befehle an den Roboter senden und Daten von ihm empfangen, was eine nahtlose Integration in das VR-Erlebnis ermöglicht.

\subsubsection{Robotersteuerung über ROS}

Pepper, der humanoide Roboter, wird über das Robot Operating System (ROS) gesteuert, das eine Standardplattform für die Entwicklung von Robotersoftware ist. ROS bietet eine Vielzahl von Funktionen zur Robotersteuerung, einschließlich der Unterstützung für verschiedene Sensoren, Aktuatoren und Navigationssysteme. Indem die ROS TCP-Verbindung verwendet wird, kann die Unity-Anwendung mit den ROS-Nodes kommunizieren, die für die Steuerung von Pepper zuständig sind. Dies ermöglicht es der Unity-Anwendung, Pepper-Bewegungen zu steuern und Sensordaten von Pepper zu empfangen, um ein interaktives VR-Erlebnis zu schaffen.

\subsubsection{Implementierung in Unity}

Die Implementierung der ROS TCP-Verbindung in Unity erfolgt mithilfe von Plugins oder eigenen Skripten, die die TCP/IP-Kommunikation ermöglichen. Durch die Verwendung von vorhandenen ROS-Bibliotheken oder der Erstellung benutzerdefinierter ROS-Nodes kann die Unity-Anwendung ROS-Nachrichten senden und empfangen, um mit Pepper zu interagieren. Dies ermöglicht es, die Bewegungen von Pepper in Echtzeit zu steuern und Feedbackdaten von Pepper in die Unity-Anwendung zu integrieren.
\\

\noindent
Insgesamt wurde die ROS TCP-Verbindung aufgrund ihrer Kompatibilität mit Unity und der Robotersteuerung über ROS als ideale Lösung für die Kommunikation zwischen der Unity-Anwendung und dem humanoiden Roboter Pepper ausgewählt. Diese Entscheidung ermöglicht es, eine immersive und interaktive VR-Erfahrung zu schaffen, bei der die Benutzer direkt mit dem Roboter interagieren können.