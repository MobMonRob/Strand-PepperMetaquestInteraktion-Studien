\chapter{Einleitung}
Im Umfeld der \ac{DHBW} Karlsruhe werden immer wieder Projekte im Bereich der Robotik an Studierende im Rahmen verschiedener Arbeiten vergeben. Ebenso dieses Projekt, welches im Rahmen der Studienarbeit an die beiden Studierenden Matthias Schuhmacher und Marlene Rieder übergeben wurde. 
=======
Im Umfeld der \ac{DHBW} Karlsruhe werden immer wieder Projekte im Bereich der Robotik an Studierende im Rahmen verschiedener Arbeiten vergeben. Ebenso dieses Projekt, welches im Rahmen der Studienarbeit an die beiden Studierenden Matthias Schuhmacher und Marlene Rieder übergeben wurde.
\\

\noindent 
In unserer immer digitaler werdenden Welt spiele sowohl Robotik als auch \ac{VR} eine immer größer werdende Rolle, diese zwei Welten sollen durch das, im folgenden beschriebene Projekt verbunden werden.
\\

\noindent
Ziel dieses Projektes ist es den humanoiden Roboter Pepper mit Hilfe der \ac{VR} Brille Meta Quest 3 zu steuern. Dies soll ermöglicht werden, indem das Kamerabild des Roboters an die Brille übertragen wird und sich die Bewegungen des Roboters durch die Controller der Brille steuern lassen. Der Nutzer soll also sehen, was sich vor dem Roboter befindet und ihn dann steuern können.

