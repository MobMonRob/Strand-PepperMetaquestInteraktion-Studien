\chapter{Einleitung}
Im Umfeld der \ac{DHBW} Karlsruhe werden immer wieder Projekte im Bereich der Robotik an Studierende im Rahmen verschiedener Arbeiten vergeben. Ebenso dieses Projekt, welches im Rahmen der Studienarbeit an die beiden Studierenden Matthias Schuhmacher und Marlene Rieder übergeben wurde. 
\\

\noindent 
In unserer immer digitaler werdenden Welt spiele sowohl Robotik als auch \ac{VR} eine immer größer werdende Rolle, diese zwei Welten sollen durch das, im folgenden beschriebene Projekt verbunden werden.
\\

\noindent
Ziel dieses Projektes ist es den humanoiden Roboter Pepper mit Hilfe der \ac{VR} Brille Meta Quest 3 zu steuern. Dies soll ermöglicht werden, indem das Kamerabild des Roboters an die Brille übertragen wird und sich die Bewegungen des Roboters durch die Controller der Brille steuern lassen. Der Nutzer soll also sehen, was sich vor dem Roboter befindet und ihn mit diesen Informationen steuern können.

\section{Motivation}
Die Steuerung eines humanoiden Roboters bietet eine Vielzahl von Vorteilen. Durch die Fernsteuerung können Roboter in gefährlichen oder unzugänglichen Umgebungen eingesetzt werden, was die Sicherheit der menschlichen Bediener erhöht. Außerdem ermöglicht die Verwendung von VR-Technologie eine intuitivere und natürlichere Steuerung der Roboter, da die Bediener das Gefühl haben, direkt vor Ort zu sein und in Echtzeit mit der Umgebung zu interagieren. Dieses und viele weitere Mögliche Einsatzszenarien haben uns motiviert an diesem Projekt zu arbeiten.

\section{Ziele und Fragestellungen}
Das Hauptziel dieser Arbeit ist es, eine funktionsfähige Verbindung zwischen einer VR-Brille und dem humanoiden Roboter Pepper herzustellen. Dabei sollen folgende Fragestellungen untersucht werden:
\begin{itemize}
    \item Wie kann die Datenübertragung zwischen der VR-Brille und dem Roboter gestaltet werden? Dies mit einer möglichst latenzfreien Steuerung?
    \item Welche Herausforderungen ergeben sich bei der Integration des Interfaces und der Kontrollsprache zur Steuerung des Roboters?
    \item Welche technischen und ergonomischen Anforderungen müssen bei der Entwicklung der VR-Steuerung berücksichtigt werden?
\end{itemize}

\section{Methodik}
Zur Beantwortung dieser Fragestellungen wurde ein methodischer Ansatz verfolgt, der sowohl theoretische als auch praktische Aspekte beinhaltet. Zunächst wurde eine umfassende Literaturrecherche durchgeführt, um den aktuellen Stand der Technik und die verfügbaren Technologien zu analysieren. Anschließend wurde sich mit der Auswahl der Technologien beschäftigt und begonnen sowohl das Interface als auch die Robotersteuerung zu implementieren.

\section{Struktur der Arbeit}
Diese Arbeit gliedert sich wie folgt:
\begin{itemize}
    \item Kapitel 2 beschreibt die theoretischen Grundlagen sowie den aktuellen Stand der Technik.
    \item Kapitel 3 erläutert die technischen Spezifikationen der verwendeten Hardware und Software und warum genau diese Elemente gewählt wurden.
    \item Kapitel 4 stellt den Entwicklungsprozess und die Implementierung der Verbindung dar.
    \item Kapitel 5 präsentiert eine Benutzerstudie zur Qualitätssicherung.
    \item Kapitel 6 zeigt in welchen Gebieten das Projekt eingesetzt werden kann.
    \item Kapitel 7 Umfasst ein kurzes Fazit über den gesamten Arbeitszeitraum
    \item Kapitel 8 Beschreibt wie das Projekt in Zukunft fortgeführt werden kann.
\end{itemize}

