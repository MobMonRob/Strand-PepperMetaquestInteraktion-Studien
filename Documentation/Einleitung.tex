\chapter{Einleitung}
Im Umfeld der \ac{DHBW} Karlsruhe werden immer wieder Projekte im Bereich der Robotik an Studierende im Rahmen verschiedener Arbeiten vergeben. Ebenso dieses Projekt, welches im Rahmen der Studienarbeit an die beiden Studierenden Matthias Schuhmacher und Marlene Rieder übergeben wurde. 
\\

\noindent
Im Umfeld der \ac{DHBW} Karlsruhe werden immer wieder Projekte im Bereich der Robotik an Studierende im Rahmen verschiedener Arbeiten vergeben. Ebenso dieses Projekt, welches im Rahmen der Studienarbeit an die beiden Studierenden Matthias Schuhmacher und Marlene Rieder übergeben wurde.
\\

\noindent 
In unserer immer digitaler werdenden Welt spiele sowohl Robotik als auch \ac{VR} eine immer größer werdende Rolle, diese zwei Welten sollen durch das, im folgenden beschriebene Projekt verbunden werden.
\\

\noindent
Ziel dieses Projektes ist es den humanoiden Roboter Pepper mit Hilfe der \ac{VR} Brille Meta Quest 3 zu steuern. Dies soll ermöglicht werden, indem das Kamerabild des Roboters an die Brille übertragen wird und sich die Bewegungen des Roboters durch die Controller der Brille steuern lassen. Der Nutzer soll also sehen, was sich vor dem Roboter befindet und ihn dann steuern können.

\section{Motivation}
Die Teleoperation humanoider Roboter bietet eine Vielzahl von Vorteilen. Durch die Fernsteuerung können Roboter in gefährlichen oder unzugänglichen Umgebungen eingesetzt werden, was die Sicherheit der menschlichen Bediener erhöht. Außerdem ermöglicht die Verwendung von VR-Technologie eine intuitivere und natürlichere Steuerung der Roboter, da die Bediener das Gefühl haben, direkt vor Ort zu sein und in Echtzeit mit der Umgebung zu interagieren.

\section{Ziele und Fragestellungen}
Das Hauptziel dieser Arbeit ist es, eine funktionsfähige Verbindung zwischen der MetaQuest 3 VR-Brille und dem humanoiden Roboter Pepper herzustellen. Dabei sollen folgende Fragestellungen untersucht werden:
\begin{itemize}
    \item Wie kann die Datenübertragung zwischen der VR-Brille und dem Roboter optimiert werden, um eine latenzfreie Steuerung zu ermöglichen?
    \item Welche Herausforderungen ergeben sich bei der Integration von Unity und ROS zur Steuerung des Roboters?
    \item Welche technischen und ergonomischen Anforderungen müssen bei der Entwicklung der VR-Steuerung berücksichtigt werden?
\end{itemize}

\section{Methodik}
Zur Beantwortung dieser Fragestellungen wird ein methodischer Ansatz verfolgt, der sowohl theoretische als auch praktische Aspekte umfasst. Zunächst wird eine umfassende Literaturrecherche durchgeführt, um den aktuellen Stand der Technik und die verfügbaren Technologien zu analysieren. Anschließend wird ein Prototyp entwickelt, der die Verbindung zwischen der VR-Brille und dem Roboter herstellt. Dieser Prototyp wird in mehreren Iterationen getestet und optimiert, um die bestmögliche Leistung zu erzielen.

\section{Struktur der Arbeit}
Diese Arbeit gliedert sich wie folgt:
\begin{itemize}
    \item Kapitel 2 beschreibt die theoretischen Grundlagen und den aktuellen Stand der Technik.
    \item Kapitel 3 erläutert die technischen Spezifikationen der verwendeten Hardware und Software.
    \item Kapitel 4 stellt den Entwicklungsprozess und die Implementierung des Prototyps dar.
    \item Kapitel 5 präsentiert die Testergebnisse und Diskussion.
    \item Kapitel 6 gibt eine Zusammenfassung und Ausblick auf zukünftige Arbeiten.
\end{itemize}

