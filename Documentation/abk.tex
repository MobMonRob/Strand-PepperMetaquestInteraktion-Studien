%%%%%%%%%%%%%%%%%%%%%%%%%%%%%%%%%%%%%%%%%%%%%%%%%%%%%%%%%%%%%%%%%%%%%%%%%%%%%%
%% Descr:       Vorlage für Berichte der DHBW-Karlsruhe, Datei mit Abkürzungen
%% Author:      Prof. Dr. Jürgen Vollmer, vollmer@dhbw-karlsruhe.de
%% $Id: abk.tex,v 1.4 2017/10/06 14:02:03 vollmer Exp $
%% -*- coding: utf-8 -*-
%%%%%%%%%%%%%%%%%%%%%%%%%%%%%%%%%%%%%%%%%%%%%%%%%%%%%%%%%%%%%%%%%%%%%%%%%%%%%%%

\chapter*{Abkürzungsverzeichnis}                   % chapter*{..} -->   keine Nummer, kein "Kapitel"
						         % Nicht ins Inhaltsverzeichnis
% \addcontentsline{toc}{chapter}{Akürzungsverzeichnis}   % Damit das doch ins Inhaltsverzeichnis kommt

% Hier werden die Abkürzungen definiert
\begin{acronym}[DHBW]
  % \acro{Name}{Darstellung der Abkürzung}{Langform der Abkürzung}
  \acro{Abk}[Abk.]{Abkürzung}
  \acro{DHBW} {Duale Hochschule Baden-Württemberg}
  \acro{VR} [VR] {Virtual Reality}
  \acro{ACM}[ACM]{Association for Computing Machinery}
  \acro{SAS}[SAS]{Société par actions simplifiée}
  \acro{JMS}[JMS]{Java Message Service}
  \acro{DDS}[DDS]{Data Distribution Service}
  \acro{MOM}[MOM]{Message Oriented Middleware}
  \acro{ROS}[ROS]{Robot Operating System}
  \acro{B2B}[B2b]{Business-to-Business}
  \acro{TH}[TH]{Technische Hochschule}
  \acro{API}[API]{Application Programming Interface}
  \acro{SDK}[SDK]{Software Development Kit}
  \acro{URL}[URL]{Uniform Resource Locator}
  \acro{IP}[IP]{Internet Protocol}
  \acro{RPC}[RPC]{Remote Procedure Call}
  \acro{LPC}[LPC]{Local Procedure Call}
  \acro{ESB}[ESB]{Enterprise Service Bus}
  \acro{IDL}[IDL]{Interface Definition Language}
  \acro{IMU}[IMU]{Inertial Measurement Unit}
  \acro{CLI}[CLI]{Command Line Interface}
  \acro{XML}[XML]{Extensible Markup Language}
  \acro{YAML}[YAML]{YAML Ain't Markup Language}
  \acro{rcl}[rcl]{ROS Client Library}
  \acro{JVM}[JVM]{Java Virtual Machine}
  \acro{SFML}[SFML]{Simple and Fast Multimedia Library}
  \acro{URDF}[URDF]{Unified Robot Description Format}
  \acro{SRDF}[SRDF]{Semantic Robot Description Format}
  \acro{AR}[AR]{Augmented Reality}
  \acro{HRI}[HRI]{Human Robot Interaction}
  \acro{JPEG}[JPEG]{Joint Photographic Experts Group}
  \acro{HD}[HD]{High Definition}
  \acro{FPS}[FPS]{Frames per Second}
  \acro{TCP}[TCP]{Transmission Control Protocol}
  % Folgendes benutzen, wenn der Plural einer Abk. benöigt wird
  % \newacroplural{Name}{Darstellung der Abkürzung}{Langform der Abkürzung}
  \newacroplural{Abk}[Abk-en]{Abkürzungen}
  % Wenn neicht benutzt, erscheint diese Abk. nicht in der Liste
  \acro{NUA}{Not Used Acronym}
\end{acronym}
