\chapter{Anwendungsgebiete}

Die Teleoperation humanoider Roboter bietet eine Vielzahl von Einsatzmöglichkeiten in unterschiedlichen Bereichen. Im Folgenden werden einige der wichtigsten Anwendungsgebiete beschrieben:

\subsection{Medizin und Pflege}
In der Medizin und Pflege können humanoide Roboter zur Unterstützung von Ärzten und Pflegepersonal eingesetzt werden. Sie können in gefährlichen oder infektiösen Umgebungen arbeiten, wodurch das Risiko für medizinisches Personal minimiert wird. Zudem können Roboter Routineaufgaben übernehmen, was die Arbeitsbelastung des Personals verringert und mehr Zeit für die direkte Patientenbetreuung ermöglicht.

\subsection{Industrielle Fertigung}
In der industriellen Fertigung können humanoide Roboter für Aufgaben eingesetzt werden, die für Menschen gefährlich oder ergonomisch ungünstig sind. Sie können in gefährlichen Umgebungen arbeiten, schwere Lasten heben oder repetitive Aufgaben mit hoher Präzision ausführen. Durch die Teleoperation können Experten aus der Ferne eingreifen und die Roboter steuern, was die Flexibilität und Effizienz der Fertigungsprozesse erhöht.

\subsection{Katastrophenhilfe und Rettungseinsätze}
Humanoide Roboter können in Katastrophengebieten eingesetzt werden, um Menschen zu retten oder gefährliche Situationen zu erkunden. Sie können durch Trümmer navigieren, Verletzte finden und erste Hilfe leisten. Durch die Teleoperation können Rettungskräfte aus sicherer Entfernung arbeiten und dennoch präzise und effektiv handeln.

\subsection{Raumfahrt}
In der Raumfahrt können humanoide Roboter für Außenbordeinsätze (EVAs) und Wartungsarbeiten an Raumstationen oder Satelliten eingesetzt werden. Sie können in extremen Umgebungen arbeiten, die für Menschen gefährlich sind, und komplexe Aufgaben mit hoher Präzision ausführen. Die Teleoperation ermöglicht es, dass die Roboter von der Erde aus gesteuert werden, wodurch das Risiko für Astronauten minimiert wird.

\subsection{Bildung und Forschung}
Humanoide Roboter können in Bildungseinrichtungen und Forschungslabors eingesetzt werden, um Experimente durchzuführen und komplexe Konzepte zu demonstrieren. Sie können als Lehrassistenten dienen und Schülern und Studenten interaktive und praxisnahe Lernerfahrungen bieten. In der Forschung können Roboter zur Untersuchung neuer Technologien und zur Entwicklung innovativer Anwendungen verwendet werden.

\subsection{Heim- und Unterhaltungselektronik}
Im Bereich der Heim- und Unterhaltungselektronik können humanoide Roboter als persönliche Assistenten oder Unterhaltungsgeräte eingesetzt werden. Sie können Aufgaben im Haushalt übernehmen, wie das Aufräumen oder das Zubereiten von Mahlzeiten, und bieten interaktive Unterhaltungsmöglichkeiten, wie das Spielen von Spielen oder das Führen von Gesprächen. Durch die Teleoperation können Benutzer ihre Roboter aus der Ferne steuern und ihnen Anweisungen geben.

\subsection{Kundendienst und Gastgewerbe}
Humanoide Roboter können im Kundendienst und im Gastgewerbe eingesetzt werden, um Kunden zu begrüßen, Informationen bereitzustellen oder Bestellungen entgegenzunehmen. Sie können in Hotels, Restaurants oder Geschäften arbeiten und den Service verbessern, indem sie rund um die Uhr verfügbar sind und in verschiedenen Sprachen kommunizieren können. Die Teleoperation ermöglicht es, dass Menschen aus der Ferne eingreifen und bei Bedarf Unterstützung leisten können.

Die Anwendungsgebiete der Teleoperation humanoider Roboter sind vielfältig und bieten zahlreiche Möglichkeiten, die Effizienz und Sicherheit in verschiedenen Bereichen zu erhöhen. Die vorliegende Arbeit trägt dazu bei, diese Vision durch die Entwicklung einer VR-basierten Steuerung für den humanoiden Roboter Pepper weiter voranzutreiben.
