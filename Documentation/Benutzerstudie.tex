\chapter{Benutzerstudie und Qualitätssicherung}
Um die Software an echten Probanden zu testen wird im folgenden der Prozess einer Benutzerstudie zur Qualitätssicherung beschrieben.
\section{Durchführung der Studie}
\begin{itemize}
   \item \textbf{Instruktion der Teilnehmer:} Bereitstellung von klaren Anweisungen und Schulungen für die Teilnehmer zur korrekten Verwendung der VR-Steuerung und zur Durchführung der Aufgaben.
   \item \textbf{Durchführung der Aufgaben:} Aufforderung der Teilnehmer, bestimmte Aufgaben mit der VR-Steuerung auszuführen, während ihre Interaktionen und Leistungen aufgezeichnet werden.
\end{itemize}

\section{Datenerfassung und -analyse}
\begin{itemize}
   \item \textbf{Quantitative Metriken:} Erfassung quantitativer Daten wie Genauigkeit, Reaktionszeit und Effizienz bei der Durchführung der Aufgaben.
   \item \textbf{Qualitative Bewertungen:} Sammlung von qualitativen Feedbacks der Teilnehmer durch Interviews, Fragebögen oder offene Diskussionen über ihre Erfahrungen und Meinungen zur Steuerung.
\end{itemize}

\section{Auswertung der Ergebnisse}
\begin{itemize}
   \item \textbf{Analyse der Daten:} Auswertung der gesammelten Daten, um Trends, Muster und statistische Signifikanz zu identifizieren.
   \item \textbf{Interpretation des Feedbacks:} Interpretation des qualitativen Feedbacks der Teilnehmer, um Einblicke in ihre Zufriedenheit, Schwierigkeiten oder Verbesserungsvorschläge zu gewinnen.
\end{itemize}

\section{Schlussfolgerungen und Empfehlungen}
\begin{itemize}
   \item \textbf{Zusammenfassung der Ergebnisse:} Präsentation der wichtigsten Ergebnisse der Studie und deren Bedeutung im Hinblick auf die gesteckten Ziele.
   \item \textbf{Ableitung von Empfehlungen:} Ableitung von Empfehlungen zur Verbesserung der VR-Steuerung basierend auf den Ergebnissen und dem Feedback der Teilnehmer.
\end{itemize}
