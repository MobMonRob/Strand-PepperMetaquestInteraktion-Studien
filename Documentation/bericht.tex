%&bericht

%%%%%%%%%%%%%%%%%%%%%%%%%%%%%%%%%%%%%%%%%%%%%%%%%%%%%%%%%%%%%%%%%%%%%%%%%%%%%%%
%% Descr:       Vorlage für Berichte der DHBW-Karlsruhe
%% Author:      Prof. Dr. Jürgen Vollmer, juergen.vollmer@dhbw-karlsruhe.de
%% $Id: bericht.tex,v 1.25 2020/03/13 15:07:45 vollmer Exp $
%%  -*- coding: utf-8 -*-
%%%%%%%%%%%%%%%%%%%%%%%%%%%%%%%%%%%%%%%%%%%%%%%%%%%%%%%%%%%%%%%%%%%%%%%%%%%%%%%

\documentclass[
   ngerman          % neue deutsche Rechtschreibung
  ,a4paper 
  ,oneside         % Papiergrösse
% ,twoside          % Zweiseitiger Druck (rechts/links)
% ,10pt             % Schriftgrösse
% ,11pt
  ,12pt
  ,pdftex
  ,parskip=half
  ,headsepline
  ,abstraction
  ,footsepline  
%  ,disable         % Todo-Markierungen auschalten
]{report}
\usepackage[ngerman]{babel}
\usepackage[T1]{fontenc}
% Bitte die Codierung Ihrer Dateien auswählen:
% \usepackage[latin1]{inputenc}    % Für UNIX mit ISO-LATIN-codierten Dateien
% \usepackage[applemac]{inputenc}  % Für Apple Mac
% \usepackage[ansinew]{inputenc}   % Für Microsoft Windows
\usepackage[utf8]{inputenc}        % UTF-8 codierte Dateien
                                   % Dieses Dokument ist unter Unix erstellt, daher
                                   % wird diese Input-Codierung benutzt.

\usepackage{bericht}


%% ACHTUNG, wenn man eine eigene Formatdatei (bericht.fmt) benutzt, werden Änderungen an bericht.sty
%% erst wirksam, wenn die Format-Datei neu erzeugt wurde!!!
%% Genauer alle Änderungen, die textuell vor der nächsten Zeile ".... endofdump...." stehen
%% werden erst wirksam, wenn die Formatdatei neu erzeugt wurde
\csname endofdump\endcsname

%%%%%%%%%%%%%%%%%%%%%%%%%%%%%%%%%%%%%%%%%%%%%%%%%%%%%%%%%%%%%%%%%%%%%%%%%%%%%%%
%% Angaben zur Arbeit
%%%%%%%%%%%%%%%%%%%%%%%%%%%%%%%%%%%%%%%%%%%%%%%%%%%%%%%%%%%%%%%%%%%%%%%%%%%%%%%

\newcommand{\Autor}{Matthias Schuhmacher und Marlene Rieder}
\newcommand{\MatrikelNummer}{4128647 und 8261867}
\newcommand{\Kursbezeichnung}{tinf21b3 und tinf21b5}

%\newcommand{\FirmenName}{Siemens AG}
%\newcommand{\FirmenStadt}{Karlsruhe}
%\newcommand{\FirmenLogoDeckblatt}{{\includegraphics[width=5cm]{Siemens_Logo_transp}}}

% Falls es kein Firmenlogo gibt:
%  \newcommand{\FirmenLogoDeckblatt}{}

%\newcommand{\BetreuerFirma}{Boris Rojs}
\newcommand{\BetreuerDHBW}{Prof. Dr. Marcus Strand}

%%%%%%%%%%%%%%%%%%%%%%%%%%%%%%%%%%%%%%%%%%%%%%%%%%%%%%%%%%%%%%%%%%%%%%%%%%%%%%%%%%%%%

% Wird auf dem Deckblatt und in der Erklärung benutzt:
%\newcommand{\Was}{Projekt-/Studien-/Bachleorarbeit}
%\newcommand{\Was}{Projektarbeit}
\newcommand{\Was}{Studienarbeit}
%\newcommand{\Was}{Bachleorarbeit}

%%%%%%%%%%%%%%%%%%%%%%%%%%%%%%%%%%%%%%%%%%%%%%%%%%%%%%%%%%%%%%%%%%%%%%%%%%%%%%%%%%%%%

\newcommand{\Titel}{Pepper VR – Teleopperation eines humanuiden Roboter auf Basis der Analyse menschlicher Bewegung}
\newcommand{\AbgabeDatum}{20. Mai 2024}

\newcommand{\Dauer}{300 Stunden}

% \newcommand{\Abschluss}{Bachelor of Engineering}
\newcommand{\Abschluss}{Bachelor of Science}

%\newcommand{\Studiengang}{Informatik / Informationstechnik}
 \newcommand{\Studiengang}{Informatik / Angewandte Informatik}

\hypersetup{%%
  pdfauthor={\Autor},
  pdftitle={\Titel},
  pdfsubject={\Was}
}

%%%%%%%%%%%%%%%%%%%%%%%%%%%%%%%%%%%%%%%%%%%%%%%%%%%%%%%%%%%%%%%%%%%%%%%%%%%%%%%

% Wenn \includeonly{..} benutzt wird, werden nur diese Kaptitel ausgegeben.
%\includeonly{
 % abk
%  Einleitung
 %,Grundlagen
 %,Die Projekte
 %,Reflexion und Ausblick
%,changelog%
%}

%%%%%%%%%%%%%%%%%%%%%%%%%%%%%%%%%%%%%%%%%%%%%%%%%%%%%%%%%%%%%%%%%%%%%%%%%%%%%%%

% Benutzt man das "biblatex"-Paket, dann muß das hier stehen:
% siehe auch die mit BIBLATEX markierten Zeilen in bericht.sty
\bibliography{T1000.bib}
\def\SymbReg{\textsuperscript{\textregistered}}
\nocite{*}
\begin{document}

%%%%%%%%%%%%%%%%%%%%%%%%%%%%%%%%%%%%%%%%%%%%%%%%%%%%%%%%%%%%%%%%%%%%%%%%%%%%%%%

\begin{titlepage}
\begin{center}
\vspace*{-2cm}
\includegraphics[width=4cm]{dhbw-logo}\\[2cm]
{\Huge \Titel}\\[1cm]
{\Huge\scshape \Was}\\[1cm]
{\large für die Prüfung zum}\\[0.5cm]
{\Large \Abschluss}\\[0.5cm]
{\large des Studienganges \Studiengang}\\[0.5cm]
{\large an der}\\[0.5cm]
{\large Dualen Hochschule Baden-Württemberg Karlsruhe}\\[0.5cm]
{\large von}\\[0.5cm]
{\large\bfseries \Autor}\\[1cm]
{\large Abgabedatum \AbgabeDatum}
\vfill
\end{center}
\begin{tabular}{l@{\hspace{2cm}}l}
Bearbeitungszeitraum	         & \Dauer 			\\
Matrikelnummer	                 & \MatrikelNummer		\\
Kurs			         & \Kursbezeichnung		\\
%Ausbildungsfirma	         & \FirmenName			\\
%			         & \FirmenStadt			\\
%Betreuer der Ausbildungsfirma	 & \BetreuerFirma		\\
Gutachter der Studienakademie	 & \BetreuerDHBW		\\
\end{tabular}
\end{titlepage}

%%%%%%%%%%%%%%%%%%%%%%%%%%%%%%%%%%%%%%%%%%%%%%%%%%%%%%%%%%%%%%%%%%%%%%%%%%%%%%%

%%%%%%%%%%%%%%%%%%%%%%%%%%%%%%%%%%%%%%%%%%%%%%%%%%%%%%%%%%%%%%%%%%%%%%%%%%%%%%%

%% Descr:       Vorlage für Berichte der DHBW-Karlsruhe, Erklärung

%% Author:      Prof. Dr. Jürgen Vollmer, vollmer@dhbw-karlsruhe.de

%% $Id: erklaerung.tex,v 1.11 2020/03/13 14:24:42 vollmer Exp $

%% -*- coding: utf-8 -*-

%%%%%%%%%%%%%%%%%%%%%%%%%%%%%%%%%%%%%%%%%%%%%%%%%%%%%%%%%%%%%%%%%%%%%%%%%%%%%%%

 

% In Bachelorarbeiten muss eine schriftliche Erklärung abgegeben werden.

% Hierin bestätigen die Studierenden, dass die Bachelorarbeit, etc.

% selbständig verfasst und sämtliche Quellen und Hilfsmittel angegeben sind. Diese Erklärung

% bildet das zweite Blatt der Arbeit. Der Text dieser Erklärung muss auf einer separaten Seite

% wie unten angegeben lauten.

 

\newpage
\thispagestyle{empty}
\begin{framed}
\begin{center}
\Large\bfseries Erklärung
\end{center}
\medskip
\noindent
% siehe §5(3) der \enquote{Studien- und Prüfungsordnung DHBW Technik} vom 29.\,9.\,2017 und Anhang 1.1.13
Ich versichere hiermit, dass ich meine \Was mit dem Thema:
\enquote{\Titel}
selbstständig verfasst und keine anderen als die angegebenen Quellen und Hilfsmittel benutzt habe. Ich versichere zudem, dass die eingereichte elektronische Fassung mit der gedruckten Fassung übereinstimmt.
\vspace{3cm}
\\
\noindent
\underline{\hspace{4cm}}\hfill\underline{\hspace{6cm}}\\
Ort~~~~~Datum\hfill Unterschrift\hspace{4cm}
\end{framed} 

\vfill
%\begin{framed}
%\begin{center}
%\Large\bfseries Sperrvermerk
%\end{center}
%\medskip
%\noindent
%Die vorliegende Projektarbeit T3\_2000 mit dem Titel
%\enquote{\Titel}
%enthält unternehmensinterne bzw. vertrauliche Informationen der Atruvia AG, ist deshalb mit einem Sperrvermerk versehen und wird ausschließlich zu Prüfungszwecken am Studiengang \Studiengang der Dualen Hochschule Baden-Württemberg Karlsruhe vorgelegt.
%\newline
%Sie ist ausschließlich zur Einsicht durch den zugeteilten Gutachter, die Leitung des Studiengangs und ggf. den Prüfungsausschuss des Studiengangs bestimmt. Es ist untersagt,
%\newline
%\indent - den Inhalt dieser Arbeit (einschließlich Daten, Abbildungen, Tabellen, Zeichnungen
%\indent   usw.) als Ganzes oder auszugsweise weiterzugeben. Kopien oder Abschriften dieser
%\indent   Arbeit (einschließlich Daten, Abbildungen, Tabellen, Zeichnungen usw.) als Ganzes
%\indent   oder auszugsweise anzufertigen.
%    \newline    
%\indent - diese Arbeit zu veröffentlichen bzw. digital, elektronisch oder virtuell zur Verfügung zu
%\indent   stellen.
 %   \newline
%Jede anderweitige Einsichtnahme und Veröffentlichung - auch von Teilen der Arbeit - bedarf der vorherigen Zustimmung durch den Verfasser und die Atruvia AG.

%\end{framed}

%%%%%%%%%%%%%%%%%%%%%%%%%%%%%%%%%%%%%%%%%%%%%%%%%%%%%%%%%%%%%%%%%%%%%%%%%%%%%%%
\endinput
%%%%%%%%%%%%%%%%%%%%%%%%%%%%%%%%%%%%%%%%%%%%%%%%%%%%%%%%%%%%%%%%%%%%%%%%%%%%%%%


%%%%%%%%%%%%%%%%%%%%%%%%%%%%%%%%%%%%%%%%%%%%%%%%%%%%%%%%%%%%%%%%%%%%%%%%%%%%%%%

\begin{abstract}
Das Ziel der vorliegenden Studienarbeit ist es, eine Verbindung zwischen einer Virtual-Reality Brille und dem humanoiden Roboter Pepper herzustellen. Das Kamerabild des Roboters soll auf der Brille angezeigt werden, ebenfalls soll es möglich sein, den Roboter mit Hilfe der Controller der Brille zu steuern.
\end{abstract}

\newpage
\tableofcontents           % Inhaltsverzeichnis hier ausgeben
%\listoffigures             % Liste der Abbildungen
%\listoftables              % Liste der Tabellen
%\lstlistoflistings         % Liste der Listings
%\listofequations           % Liste der Formeln

% Jetzt kommt der "eigentliche" Text
%%%%%%%%%%%%%%%%%%%%%%%%%%%%%%%%%%%%%%%%%%%%%%%%%%%%%%%%%%%%%%%%%%%%%%%%%%%%%%%
%% Descr:       Vorlage für Berichte der DHBW-Karlsruhe, Datei mit Abkürzungen
%% Author:      Prof. Dr. Jürgen Vollmer, vollmer@dhbw-karlsruhe.de
%% $Id: abk.tex,v 1.4 2017/10/06 14:02:03 vollmer Exp $
%% -*- coding: utf-8 -*-
%%%%%%%%%%%%%%%%%%%%%%%%%%%%%%%%%%%%%%%%%%%%%%%%%%%%%%%%%%%%%%%%%%%%%%%%%%%%%%%

\chapter*{Abkürzungsverzeichnis}                   % chapter*{..} -->   keine Nummer, kein "Kapitel"
						         % Nicht ins Inhaltsverzeichnis
% \addcontentsline{toc}{chapter}{Akürzungsverzeichnis}   % Damit das doch ins Inhaltsverzeichnis kommt

% Hier werden die Abkürzungen definiert
\begin{acronym}[DHBW]
  % \acro{Name}{Darstellung der Abkürzung}{Langform der Abkürzung}
  \acro{Abk}[Abk.]{Abkürzung}
  \acro{DHBW} {Duale Hochschule Baden-Württemberg}
  \acro{SAS}[SAS]{Société par actions simplifiée}
  \acro{B2B}[B2b]{Business-to-Business}
  \acro{TH}[TH]{Technische Hochschule}
  \acro{API}[API]{Application Programming Interface}
  \acro{SDK}[SDK]{Software Development Kit}
  \acro{URL}[URL]{Uniform Resource Locator}
  \acro{IP}[IP]{Internet Protocol}
  \acro{RPC}[RPC]{Remote Procedure Call}
  \acro{LPC}[LPC]{Local Procedure Call}
  
  % Folgendes benutzen, wenn der Plural einer Abk. benöigt wird
  % \newacroplural{Name}{Darstellung der Abkürzung}{Langform der Abkürzung}
  \newacroplural{Abk}[Abk-en]{Abkürzungen}
  % Wenn neicht benutzt, erscheint diese Abk. nicht in der Liste
  \acro{NUA}{Not Used Acronym}
\end{acronym}
              % Abkürzungsverzeichnis
\include{einleitung}
\include{grundlagen}
\include{technologieauswahl}
\include{umsetzung}
\include{fazit}
\include{fortsetzung}


% Ab hier beginnt der Anhang
\appendix
\addcontentsline{toc}{chapter}{Literaturverzeichnis}

% Haben Sie das "biblatex"-Paket nicht installiert, benutzen Sie folgendes:
% Ohne das "biblatex"-Paket (s. bericht.sty) produziert folgendes
% "deutsche" Zitate in Literaturverzeichnissen gemaß der Norm DIN 1505,
% Teil 2 vom Jan. 1984.
% Die Zitatmarken werden alphabetisch nach Verfassern
% sortiert und sind durch abgekürzte Verfasserbuchstaben plus
% Erscheinungsjahr in eckigen Klammern gekennzeichnet.

% \bibliographystyle{alphadin}
% \bibliography{bericht}

%%%%%%%%%%%%%%%%%%%%%%%%%%%%%%%%%%%%%%%5
% BIBLATEX
% Benutzt man das "biblatex"-Paket, muß man folgendes schreiben:
\def\refname{Literaturverzeichnis}
\printbibliography
%%%%%%%%%%%%%%%%%%%%%%%%%%%%%%%%%%%%%%%5


\end{document}
